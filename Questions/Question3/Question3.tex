\documentclass[11pt,preprint, authoryear]{elsarticle}

\usepackage{lmodern}
%%%% My spacing
\usepackage{setspace}
\setstretch{1.2}
\DeclareMathSizes{12}{14}{10}{10}

% Wrap around which gives all figures included the [H] command, or places it "here". This can be tedious to code in Rmarkdown.
\usepackage{float}
\let\origfigure\figure
\let\endorigfigure\endfigure
\renewenvironment{figure}[1][2] {
    \expandafter\origfigure\expandafter[H]
} {
    \endorigfigure
}

\let\origtable\table
\let\endorigtable\endtable
\renewenvironment{table}[1][2] {
    \expandafter\origtable\expandafter[H]
} {
    \endorigtable
}


\usepackage{ifxetex,ifluatex}
\usepackage{fixltx2e} % provides \textsubscript
\ifnum 0\ifxetex 1\fi\ifluatex 1\fi=0 % if pdftex
  \usepackage[T1]{fontenc}
  \usepackage[utf8]{inputenc}
\else % if luatex or xelatex
  \ifxetex
    \usepackage{mathspec}
    \usepackage{xltxtra,xunicode}
  \else
    \usepackage{fontspec}
  \fi
  \defaultfontfeatures{Mapping=tex-text,Scale=MatchLowercase}
  \newcommand{\euro}{€}
\fi

\usepackage{amssymb, amsmath, amsthm, amsfonts}

\def\bibsection{\section*{References}} %%% Make "References" appear before bibliography


\usepackage[round]{natbib}

\usepackage{longtable}
\usepackage[margin=2.3cm,bottom=2cm,top=2.5cm, includefoot]{geometry}
\usepackage{fancyhdr}
\usepackage[bottom, hang, flushmargin]{footmisc}
\usepackage{graphicx}
\numberwithin{equation}{section}
\numberwithin{figure}{section}
\numberwithin{table}{section}
\setlength{\parindent}{0cm}
\setlength{\parskip}{1.3ex plus 0.5ex minus 0.3ex}
\usepackage{textcomp}
\renewcommand{\headrulewidth}{0.2pt}
\renewcommand{\footrulewidth}{0.3pt}

\usepackage{array}
\newcolumntype{x}[1]{>{\centering\arraybackslash\hspace{0pt}}p{#1}}

%%%%  Remove the "preprint submitted to" part. Don't worry about this either, it just looks better without it:
\makeatletter
\def\ps@pprintTitle{%
  \let\@oddhead\@empty
  \let\@evenhead\@empty
  \let\@oddfoot\@empty
  \let\@evenfoot\@oddfoot
}
\makeatother

 \def\tightlist{} % This allows for subbullets!

\usepackage{hyperref}
\hypersetup{breaklinks=true,
            bookmarks=true,
            colorlinks=true,
            citecolor=blue,
            urlcolor=blue,
            linkcolor=blue,
            pdfborder={0 0 0}}


% The following packages allow huxtable to work:
\usepackage{siunitx}
\usepackage{multirow}
\usepackage{hhline}
\usepackage{calc}
\usepackage{tabularx}
\usepackage{booktabs}
\usepackage{caption}


\newenvironment{columns}[1][]{}{}

\newenvironment{column}[1]{\begin{minipage}{#1}\ignorespaces}{%
\end{minipage}
\ifhmode\unskip\fi
\aftergroup\useignorespacesandallpars}

\def\useignorespacesandallpars#1\ignorespaces\fi{%
#1\fi\ignorespacesandallpars}

\makeatletter
\def\ignorespacesandallpars{%
  \@ifnextchar\par
    {\expandafter\ignorespacesandallpars\@gobble}%
    {}%
}
\makeatother

\newlength{\cslhangindent}
\setlength{\cslhangindent}{1.5em}
\newenvironment{CSLReferences}%
  {\setlength{\parindent}{0pt}%
  \everypar{\setlength{\hangindent}{\cslhangindent}}\ignorespaces}%
  {\par}


\urlstyle{same}  % don't use monospace font for urls
\setlength{\parindent}{0pt}
\setlength{\parskip}{6pt plus 2pt minus 1pt}
\setlength{\emergencystretch}{3em}  % prevent overfull lines
\setcounter{secnumdepth}{5}

%%% Use protect on footnotes to avoid problems with footnotes in titles
\let\rmarkdownfootnote\footnote%
\def\footnote{\protect\rmarkdownfootnote}
\IfFileExists{upquote.sty}{\usepackage{upquote}}{}

%%% Include extra packages specified by user

%%% Hard setting column skips for reports - this ensures greater consistency and control over the length settings in the document.
%% page layout
%% paragraphs
\setlength{\baselineskip}{12pt plus 0pt minus 0pt}
\setlength{\parskip}{12pt plus 0pt minus 0pt}
\setlength{\parindent}{0pt plus 0pt minus 0pt}
%% floats
\setlength{\floatsep}{12pt plus 0 pt minus 0pt}
\setlength{\textfloatsep}{20pt plus 0pt minus 0pt}
\setlength{\intextsep}{14pt plus 0pt minus 0pt}
\setlength{\dbltextfloatsep}{20pt plus 0pt minus 0pt}
\setlength{\dblfloatsep}{14pt plus 0pt minus 0pt}
%% maths
\setlength{\abovedisplayskip}{12pt plus 0pt minus 0pt}
\setlength{\belowdisplayskip}{12pt plus 0pt minus 0pt}
%% lists
\setlength{\topsep}{10pt plus 0pt minus 0pt}
\setlength{\partopsep}{3pt plus 0pt minus 0pt}
\setlength{\itemsep}{5pt plus 0pt minus 0pt}
\setlength{\labelsep}{8mm plus 0mm minus 0mm}
\setlength{\parsep}{\the\parskip}
\setlength{\listparindent}{\the\parindent}
%% verbatim
\setlength{\fboxsep}{5pt plus 0pt minus 0pt}



\begin{document}



\begin{frontmatter}  %

\title{Question 3}

% Set to FALSE if wanting to remove title (for submission)




\author[Add1]{Joshua Strydom\footnote{\textbf{Contributions:}
  \newline \emph{The authors would like to thank no institution for
  money donated to this project. Thank you sincerely.}}}
\ead{20718284@sun.ac.za}





\address[Add1]{Stellenbosch University, Stellenbosch, South Africa}



\vspace{1cm}





\vspace{0.5cm}

\end{frontmatter}



%________________________
% Header and Footers
%%%%%%%%%%%%%%%%%%%%%%%%%%%%%%%%%
\pagestyle{fancy}
\chead{}
\rhead{}
\lfoot{}
\rfoot{\footnotesize Page \thepage}
\lhead{}
%\rfoot{\footnotesize Page \thepage } % "e.g. Page 2"
\cfoot{}

%\setlength\headheight{30pt}
%%%%%%%%%%%%%%%%%%%%%%%%%%%%%%%%%
%________________________

\headsep 35pt % So that header does not go over title




\hypertarget{alsi-and-swix}{%
\section{\texorpdfstring{ALSI and SWIX
\label{ALSISWIX}}{ALSI and SWIX }}\label{alsi-and-swix}}

\begin{figure}[H]

{\centering \includegraphics{Question3_files/figure-latex/Figure1-1} 

}

\caption{Cumulative returns \label{Figure1}}\label{fig:Figure1}
\end{figure}

The ALSI and the SWIX have, over the past decade, followed each other
closely in terms of cumulative returns. The SWIX seemed to have
consistently outperformed the ALSI from approximately 2011 to 2020.
Post-2020, the ALSI has outperformed the SWIX.

\begin{figure}[H]

{\centering \includegraphics{Question3_files/figure-latex/Figure2-1} 

}

\caption{Rolling returns \label{Figure2}}\label{fig:Figure2}
\end{figure}

Figures \ref{Figure2} displays the returns generated by the ALSI and
SWIX over a 6 month, 1 year, 3 year and 5 year period. The only time
that the SWIX outperformed the ALSI was over the 5 year period. The
outperformance, however, was by a meagre 0.29\%. The longer an
individual intends for their money to be in the market, the more
attractive the SWIX becomes in a relative sense.

\hypertarget{large-mid-and-small-cap-contributions-to-returns}{%
\section{Large, Mid and Small cap contributions to
returns}\label{large-mid-and-small-cap-contributions-to-returns}}

\begin{figure}[H]

{\centering \includegraphics{Question3_files/figure-latex/Figure3-1} 

}

\caption{Large caps \label{Figure3}}\label{fig:Figure3}
\end{figure}

Seeing as though the SWIX and ALSI are predominantly made up of large
cap stocks, the choice of such stocks is very important. The
contribution to cumulative returns by large cap stocks has been
historically larger for the SWIX than for the ALSI. This is not,
however, the case post-2020 as the cumulative returns for the ALSI are
larger than that of the SWIX.

\begin{figure}[H]

{\centering \includegraphics{Question3_files/figure-latex/Figure4-1} 

}

\caption{Mid caps \label{Figure4}}\label{fig:Figure4}
\end{figure}

Mid cap cumulative returns for both indices have tracked each other
reasonably well, with the ALSI generally just outperforming the SWIX.
The actual contribution of these stocks to overall cumulative returns
have not been that high though.

\begin{figure}[H]

{\centering \includegraphics{Question3_files/figure-latex/Figure5-1} 

}

\caption{Small caps \label{Figure5}}\label{fig:Figure5}
\end{figure}

The contribution of small cap stocks to cumulative returns is at best
marginal. Small cap stocks within the SWIX seem to be `better' than
those of the ALSI. The sample does not extend past 2018 though.

\hypertarget{conclusion}{%
\section{Conclusion}\label{conclusion}}

The ALSI and SWIX have largely performed equally well. The performance
of different size indexes, sector exposures and stock concentration over
time are very important.

\bibliography{Tex/ref}





\end{document}
